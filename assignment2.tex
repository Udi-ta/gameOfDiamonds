\documentclass[12 pt]{report}
\usepackage{hyperref}

\begin{document}
\title{Game of Diamonds}
\maketitle
I used perplexity genAI chatbot for this purpose. it was generally understanding what I \\
told but at the same time I was dicey of the fact if it was actually understanding the game or not.\\

\chapter{PHASE 1 - Teaching}

\section{Rules of the game}
\subsection{Setup:} 
For 2 players: Set aside the Club suit. \\
For 3 players: Shuffle all suits. \\
Place Diamonds in the center, distribute other suits to players. \\
\subsection{Bidding Process:}
Draw a random Diamond card as the lowest bid. \\
Players secretly choose bids. \\
Reveal bids; higher bid wins. \\
\subsection{Gameplay:}
Players aim to predict and win bid tricks. \\
Points: 10 per bid trick, 1 per overtrick, -10 per undertrick. \\
Start with lowest Club card; highest card of played suit wins unless a Diamond is played. \\
\subsection{Scoring:}
Score based on bids and tricks taken. \\
Winning bid determines the trump suit for that round. \\
\subsection{Winning:}
Player with the most points at the end wins. \\

\section{promts given}
All the promts given and the conversation are given in the hyperlink provided \\
\href{https://www.perplexity.ai/search/do-you-know-gHKgfZmeSsy.cOMl1H1eSQ}{Conversation with perplexity}

\section{strategies discussed}
I thought that perplexity was not actually getting the game but the strategy it gave was almost \\
mind blowing for me due the flexibility and the tricks it observed by playing the game with me.\\
It understood my strategy due to which it lost the game, and observed the undertricks where it lagged \\
thus giving a considerably nice game strategy, The srategy it gave is as follows:
\subsection{Strategic Bidding:}
Accurately predict the number of tricks you can win based on your hand strength. \\
Avoid overbidding to prevent penalties for undertricks. \\
Consider opponents' bids and adjust your bid accordingly. \\
\subsection{Trump Management:}
here trump means - suit of highest importance i.e. Diamonds here \\ 
Utilize Diamonds as trump effectively to win tricks. \\
Save high-ranking Diamond cards for crucial moments to secure bid tricks. \\
\subsection{Card Play Tactics:}
Lead with low Clubs to start tricks strategically. \\
Follow suit whenever possible, reserving high cards for key moments. \\
Use Diamonds wisely, especially when they become trump. \\
\subsection{Scoring Awareness:}
Aim to fulfill your bid while maximizing overtricks for extra points. \\
Minimize undertricks to avoid losing points. \\
\subsection{Adaptability and Observation:}
Observe opponents' plays and adjust your strategy accordingly. \\
Stay flexible during gameplay, adapting to changing circumstances. \\

\section{Code for strategy}
I gave the prompts and on basis of them it generated a code which was not fully functional,\\
 it was important to make input statements in it, again the code it provided is there in the chat\\
\href{https://www.perplexity.ai/search/do-you-know-gHKgfZmeSsy.cOMl1H1eSQ}{click here for chat link} \\
and the colab notebook where I tried testing the code which I wasn't able to complete is also here \\
for your reference \href{https://colab.research.google.com/drive/1tZvJel6zhkuNIAAIm-4oVlGKxWTS50MF?usp=sharing}{Click here for colab} \\


\chapter{PHASE 2 - Reflections}

\section{Reflections on conversations}
In my journey of teaching perplexity to GenAI, I not only devised a game strategy but also developed \\
 a code for autonomous gameplay. While perplexity's understanding fell short compared to ChatGPT, its\\
 rapid learning capabilities proved invaluable. Despite its limitations, it provided me with a wealth\\
   of insights, particularly in scenarios demanding a quick learning curve. Its proficiency in learning\\
    through action rather than comprehension was evident, highlighting its unique strengths. \\
This experience underscored the importance of leveraging diverse AI approaches, each offering distinct\\
 advantages. Through this collaboration, I gained a deeper appreciation for the multifaceted nature of \\
 artificial intelligence and its potential applications. \\
(you must know the fact that I used ChatGPT to generate this summary...)

\section{Reflections on code generated}
The generated code fell short of expectations, requiring numerous modifications despite initial \\
adjustments. Unlike Gemini, which showcased superior adaptability in code generation, this iteration \\
lacked similar proficiency. It highlighted the importance of robust AI models capable of evolving and \\
adapting to diverse tasks effectively.

\section{conclusion}
Thus This is how I taught perplexity GenAI about this game and thus developed a game strategy as well as \\
a code for playing it with computer itself. though I felt perplexity was not as good as even chatGPT at understanding \\
still it provided me with a considerable amount of reasons to use it at some places where I need \\
a faster learning rate because I felt that somehow, it is better at learning by doing not by understanding.\\

\end{document}
